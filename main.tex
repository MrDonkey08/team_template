\documentclass[10pt, a4paper]{article} % template format

% Language and font encodings
\usepackage[spanish]{babel}
\usepackage[utf8]{inputenc}
\usepackage[T1]{fontenc}
\usepackage{times} % Times New Roman
\usepackage{courier}

%% Sets page size and margins
\usepackage[margin=2.5cm, includefoot]{geometry}
%\setlength{\columnsep}{0.17in} % page columns separation

%% Useful packages
\usepackage{amsmath}
\usepackage{array} % <-- add this line for m{} column type
\usepackage[hidelinks]{hyperref} % hyperlinks support
\usepackage{graphicx} % images support
%\usepackage{listings} % codeblock support
%\usepackage{smartdiagram} % diagrams support
\usepackage[most]{tcolorbox} % callouts support
%\usepackage[colorinlistoftodos]{todonotes}
\usepackage[dvipsnames, table, xcdraw]{xcolor} % Tables support
%\usepackage{zed-csp} % cchemas support

%% Formating
\usepackage{authblk} % to add authors in maketitle
%\usepackage{blindtext} % to gen filler text
\usepackage[figurename=Fig.]{caption} % to change prefix of the image caption
\usepackage{cite} % useful to compress multiple quotations into a single entry
\usepackage{enumitem}
\usepackage{fancyhdr} % to set page style
\usepackage{indentfirst}
%\usepackage{natbib}
\usepackage{parskip} % remove first line tabulation
\usepackage{setspace}
\usepackage{titlesec}
%\usepackage{titling} % to config maketitle

%% Variables
% Main images
\newcommand{\logoUdg}{logo-udg.jpg}
\newcommand{\logoCucei}{logo-cucei.jpg}

% School data
\newcommand{\universidad}{Universidad de Guadalajara}
\newcommand{\cede}{Centro Universitario de Ciencias Exactas e Ingenierías}

% Subject data
\newcommand{\materia}{Seminario de Integración: Protocolos}
\newcommand{\carrera}{Ingeniería en Computación}
\newcommand{\division}{División de Tecnologías para la Integración CiberHumana}
\newcommand{\theTitle}{Título del Proyecto}
\newcommand{\profesor}{Profesor}
\newcommand{\seccion}{Sección}
\newcommand{\nrc}{NRC}
\newcommand{\clave}{Clave}
\newcommand{\startDate}{21 de mayo de 2024}

% Author(s) data
\newcommand{\theAuthor}{Nombre del líder del proyecto}
\newcommand{\bAuthor}{nombre del segundo participante}
\newcommand{\cAuthor}{nombre del tercer participante}
\newcommand{\dAuthor}{nombre del asesor}

\newcommand{\theAuthorCode}{A code}
\newcommand{\bAuthorCode}{B code}
\newcommand{\cAuthorCode}{C code}
\newcommand{\dAuthorCode}{D code}

\newcommand{\theAuthorMail}{first@mail.com}
\newcommand{\bAuthorMail}{second@mail.com}
\newcommand{\cAuthorMail}{third@mail.com}
\newcommand{\dAuthorMail}{fourth@mail.com}

% Authors titles
\newcommand{\aTitutlo}{Ingeniero en Computación}
\newcommand{\btitulo}{Ingeniero en Computación y Electrónica}
\newcommand{\cTitulo}{Ingeniero en Biomédica}

% Repo data
\newcommand{\repositorio}{\url{https://github.com/Forajido24/ProyectoFinal}}
\newcommand{\version}{1.0}
\newcommand{\licencia}{Sin Licencia}

%% Declaration
\date{}
\graphicspath{ {img/} }
\addto\captionsspanish{\renewcommand{\contentsname}{Índice}}
\renewcommand{\lstlistingname}{Código} % to change prefix of the code caption
\renewcommand{\lstlistlistingname}{Índice de códigos} % to change listings index title

%% Styles

% Color declaration
\definecolor{greenPortada}{HTML}{69A84F}
\definecolor{LightGray}{gray}{0.9}
\definecolor{codegreen}{rgb}{0, 0.6, 0}
\definecolor{codegray}{rgb}{0.5, 0.5, 0.5}
\definecolor{codepurple}{rgb}{0.58, 0, 0.82}
\definecolor{backcolour}{rgb}{0.95, 0.95, 0.92}

% Hyperlinks
\hypersetup{
	colorlinks=true,
	linkcolor=black,
	filecolor=greenPortada,
	urlcolor=greenPortada,
	pdfpagemode=FullScreen,
}

\urlstyle{same}

% Codeblocks
\lstdefinestyle{mystyle}{
	backgroundcolor=\color{backcolour},
	commentstyle=\color{codegreen},
	keywordstyle=\color{magenta},
	numberstyle=\tiny\color{codegray},
	stringstyle=\color{codepurple},
	basicstyle=\ttfamily\footnotesize,
	breakatwhitespace=false,
	breaklines=true,
	captionpos=b,
	keepspaces=true,
	numbers=left,
	numbersep=5pt,
	showspaces=false,
	showstringspaces=false,
	showtabs=false,
	tabsize=2
}

\lstset{style=mystyle}

%% Spacing
\newcommand{\nl}{\par\vspace{0.4cm}}
\renewcommand{\baselinestretch}{1.15} % Espaciado de línea anterior
\setlength{\parskip}{6pt} % Espaciado de línea anterior
\setlength{\parindent}{0pt} % Sangría

% Header and footer
\pagestyle{fancy}
\fancyhf{}
\renewcommand{\headrulewidth}{3pt}
\renewcommand{\headrule}{\hbox to\headwidth{\color{greenPortada}\leaders\hrule height \headrulewidth\hfill}}
\setlength{\headheight}{50pt} % Ajuste necesario para evitar warnings

% Header
\pagestyle{fancy}
\fancyhf{}
\lhead{
	\begin{minipage}[c][2cm][c]{1.3cm}
		\begin{flushleft}
			\includegraphics[width=5cm, height=1.4cm, keepaspectratio]{\logoUdg}
		\end{flushleft}
	\end{minipage}
	\begin{minipage}[c][2cm][c]{0.5\textwidth} % Adjust the height as needed
		\begin{flushleft}	
		{\materia}
		\end{flushleft}
	\end{minipage}
}
\rhead{
		\begin{minipage}[c][2cm][c]{0.4\textwidth} % Adjust the height as needed
			\begin{flushright}
				{\theTitle}
			\end{flushright}
		\end{minipage}
		\begin{minipage}[c][2cm][c]{1.3cm}
			\begin{flushright}
				\includegraphics[width=5cm, height=1.4cm, keepaspectratio]{\logoCucei}
			\end{flushright}
		\end{minipage}
}

% Footer
\fancyfoot{}
\lfoot{\small\materia}
\cfoot{\thepage} % Paginación
\rfoot{\small Curso impartido por \profesor}

%% Title
\title{
	\vspace*{-3cm}
	\fontsize{24}{28.8}\selectfont \theTitle
}
\author{\theAuthor}
\author{\bAuthor}
\author{\cAuthor}
\author{\dAuthor}

\affil{\small
	\textit{CENTRO UNIVERSITARIO DE CIENCIAS}\\
	\textit{EXACTAS E INGENIERÍAS, (CUCEI, UDG)}
}

\affil{
	\fontfamily{pcr}\selectfont
	\theAuthorMail\\
	\fontfamily{pcr}\selectfont
	\bAuthorMail\\
	\fontfamily{pcr}\selectfont
	\cAuthorMail\\
	\fontfamily{pcr}\selectfont
	\dAuthorMail
	\vspace*{-24pt}
}
\affil{}

\begin{document}
\setstretch{1} % Interlineado

\begin{titlepage}
	\newgeometry{margin=2.5cm, left=3cm, right=3cm} % change margin
	\centering
	%\vspace*{-2cm}
	{\huge\textbf{\universidad}}\par\vspace{0.6cm}
	{\LARGE{\cede}}\vfill

	\begin{figure}[h]
		\begin{minipage}[t]{0.45\textwidth}
			\centering
			\includegraphics[width=130px, height=160px, keepaspectratio]{\logoUdg}
		\end{minipage}
		\hfill
		\begin{minipage}[t]{0.45\textwidth}
			\centering
			\includegraphics[width=130px, height=160px, keepaspectratio]{\logoCucei}
		\end{minipage}
	\end{figure}\vfill


	\Large{
		\division\vfill
		\textbf{\carrera}\vfill
		\textbf{\materia}\par\vspace{3pt}
		\seccion\ - \clave\ - \nrc\vfill
	}

	\begin{figure}[h]
		\centering
		\begin{minipage}[t]{0.75\textwidth}
			{\Large
				\textbf{Integrantes:}\par\vspace{8pt}
				\begin{itemize}
					\item \theAuthor\ - \theAuthorCode
					\item \bAuthor\ - \bAuthorCode
					\item \cAuthor\ - \cAuthorCode
					\item \dAuthor\ - \dAuthorCode
				\end{itemize}
			}
		\end{minipage}
	\end{figure}\vfill

	{\LARGE{\textbf{\theTitle}}}\vfill
	
	\begin{tcolorbox}[colback=red!5!white, colframe=red!75!black]
		\centering
		Este documento contiene información sensible.\\
		No debería ser impreso o compartido con terceras entidades.
	\end{tcolorbox}\vfill
	{\large \startDate}\par
\end{titlepage}

\restoregeometry % end changed margin

%% Indexes
\clearpage
\tableofcontents

\clearpage
\listoffigures

\clearpage
\listoftables

\clearpage
\lstlistoflistings

%% Main Title
\maketitle

%% Content
This is an example LaTeX document showcasing various elements.

\section{Images}
\begin{figure}[h]
	\centering
	\includegraphics[width=0.5\textwidth]{example-image-a}
	\caption{Example Image}
	\label{fig:example}
\end{figure}

\section{Tables}
\begin{table}[h]
	\centering
	\begin{tabular}{|c|c|}
		\hline
		Column 1 & Column 2 \\
		\hline
		Item 1 & Description 1 \\
		Item 2 & Description 2 \\
		\hline
	\end{tabular}
	\caption{Example Table}
	\label{tab:example}
\end{table}

\section{Enumerations}
\begin{enumerate}
	\item First item
	\item Second item
\end{enumerate}

\section{Lists}
\begin{itemize}
	\item Item 1
	\item Item 2
\end{itemize}

\section{Equations}
\begin{equation}
	\label{eq:example}
	E = mc^2
\end{equation}

%% References

\nocite{*} % to include uncited references of .bib file

\clearpage
\bibliographystyle{ieeetr}

% generated from .bib file
\bibliography{ref}

% Manual
\iffalse
\begin{thebibliography}{9}
	\bibitem{Ejemplo1} Autor1, A. (2022). Título del libro. Editorial.
\end{thebibliography}
\fi

\end{document}
